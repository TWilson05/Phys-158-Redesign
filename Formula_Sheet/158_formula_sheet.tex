%Preamble
\documentclass{article}
\usepackage{hyperref}
\usepackage[
    type={CC},
    modifier={by-nc-sa},
    version={3.0},
]{doclicense}
\usepackage[landscape]{geometry}
\usepackage{url}
\usepackage{multicol}
\usepackage{amsmath}
\usepackage{esint}
\usepackage{amsfonts}
\usepackage{tikz}
\usetikzlibrary{decorations.pathmorphing}
\usepackage{amsmath,amssymb}

\usepackage{colortbl}
\usepackage{xcolor}
\usepackage{mathtools}
\usepackage{amsmath,amssymb}
\usepackage{enumitem}
\makeatletter

\newcommand*\bigcdot{\mathpalette\bigcdot@{.5}}
\newcommand*\bigcdot@[2]{\mathbin{\vcenter{\hbox{\scalebox{#2}{$\m@th#1\bullet$}}}}}
\makeatother

\title{Physics 158 Formula Sheet}
\usepackage[english]{babel}
\usepackage[utf8]{inputenc}

\renewcommand{\baselinestretch}{1.15}

% \advance\topmargin-.8in
% \advance\textheight3in
% \advance\textwidth3in
% \advance\oddsidemargin-1.5in
% \advance\evensidemargin-1.5in
\geometry{
 left=5mm,
 right=5mm,
 top=20mm,
 bottom=5mm,
%  showframe
}
\parindent0pt
\parskip2pt
\newcommand{\hr}{\centerline{\rule{3.5in}{1pt}}}
\newcommand\subtopic[1]{
    \textbf{#1}
    \vspace{.1cm}
    \hrule
    \vspace{.2cm}}

% custom commands
\newcommand{\ds}{\displaystyle}

\begin{document}

\begin{center}{\huge{\textbf{Physics 158 Formula Sheet}}}\\
\end{center}
\begin{multicols*}{3}

\tikzstyle{mybox} = [draw=black, fill=white, very thick,
    rectangle, rounded corners, inner sep=10pt, inner ysep=10pt]
\tikzstyle{fancytitle} =[fill=black, text=white, font=\bfseries]

% change to make vertical spacing larger or smaller
\renewcommand{\arraystretch}{1.25}



%------------ Constants ---------------
\begin{tikzpicture}
    \node [mybox] (box){%
        \begin{minipage}{0.3\textwidth}
        \small{
        \begin{tabular}{lp{8cm} l}
            Coulomb's Constant & $k=\dfrac{1}{4\pi\epsilon_0}\approx8.988\times10^9\, \mathrm{Nm^2/C^2}$\\
            Electric Constant & $\epsilon_0=8.854\times10^{-12}\, \mathrm{\frac{C^2}{Nm^2}}$\\
            Elementary Charge & $e=-1.602\times10^{-19}\, \mathrm{C}$\\
            Vacuum Permeability & $\mu_0=4\pi\times10^{-7}\, \mathrm{N/A^2}$\\
            Speed of Light & $c=\dfrac{1}{\sqrt{\epsilon_0\mu_0}}=2.998\times10^8\, \mathrm{m/s}$\\
        \end{tabular}
        }
        \end{minipage}
    };
    %------------ Constants ---------------------
    \node[fancytitle, right=10pt] at (box.north west) {Constants};
    \end{tikzpicture}

%------------ DC Circuits ---------------
\begin{tikzpicture}
\node [mybox] (box){%
    \begin{minipage}{0.3\textwidth}
        \subtopic{Resistor Circuits}
    \small{
    \begin{tabular}{lp{8cm} l}
        Ohm's Law & $V=IR$\\
        Power Dissipated & $P=IV=I^2R=\dfrac{V^2}{R}$\\
        Resistors in Series & $R_{eq}=R_1+R_2+\cdots$\\
        Resistors in Parallel & $\dfrac{1}{R_{eq}}=\dfrac{1}{R_1}+\dfrac{1}{R_2}+\cdots$\\
    \end{tabular}
    }
    \small{
    \subtopic{RC Circuits}
    \begin{tabular}{lp{8cm} l}
        Time Constant & $\tau=RC$\\
        Charging & $q(t)=Q_\text{max}(1-e^{-t/\tau})$\\
        Discharging & $q(t)=Q_\text{max}e^{-t/\tau}$\\
    \end{tabular}
    }
    \subtopic{RL Circuits}
    \small{
    \begin{tabular}{lp{8cm} l}
        Time Constant & $\tau=\dfrac{L}{R}$\\
        Charging & $i(t)=I_0(1-e^{-t/\tau})$\\
        Discharging & $i(t)=I_0e^{-t/\tau}$\\
    \end{tabular}
    }
    \subtopic{RLC Circuits}
    \small{
    \begin{tabular}{lp{8cm} l}
        Time Constant & $\tau=\dfrac{2L}{R}$\\
        Resonance Frequency & $\omega_0=\dfrac{1}{\sqrt{LC}}$\\
        Frequency & $\ds\omega=\sqrt{\omega_0^2-\frac{R^2}{4L^2}}$\\
        Charge & $q(t)=Q_\text{max}e^{-t/\tau}\cos(\omega t+\phi)$\\
    \end{tabular}
    }
    \end{minipage}
};
%------------ DC Circuits ---------------------
\node[fancytitle, right=10pt] at (box.north west) {DC Circuits};
\end{tikzpicture}

%------------ AC Circuits ---------------
\begin{tikzpicture}
    \node [mybox] (box){%
        \begin{minipage}{0.3\textwidth}
            \subtopic{Reactance and Impedance}
        \small{
        \begin{tabular}{lp{8cm} l}
            Capacitor Reactance & $X_C=\dfrac{1}{\omega C}$\\
            Capacitor Voltage & $V_C=X_CI$\\
            Inductor Reactance & $X_L=\omega L$\\
            Inductor Voltage & $V_L=X_LI$\\
            Impedance (in Series) & $|Z|^2=R^2+(X_L-X_C)^2$\\
            Voltage & $V=IZ$
        \end{tabular}
        }
        \subtopic{Phase Angles}
        \small{
        \begin{tabular}{lp{8cm} l}
            Phase Angle & $\tan\phi=\dfrac{X_L-X_C}{R}$\\
             & $\phi=\arg(v)-\arg(i)$\\
            If $v(t)=V_0\cos(\omega t)$ & then $i(t)=I_\text{max}\cos(\omega t-\phi)$
        \end{tabular}
        }
        \subtopic{Power}
        \small{
        \begin{tabular}{lp{8cm} l}
            Power Factor & $\cos\phi=\dfrac{R}{Z}$\\
            Average Power & $P_\text{avg}=V_\text{RMS}I_\text{RMS}\cos\phi=I_\text{RMS}^2R$\\
            RMS Current & $I_\text{RMS}=\dfrac{I_\text{max}}{\sqrt{2}}$\\
        \end{tabular}
        }
        \end{minipage}
    };
    %------------ AC Circuits ---------------------
    \node[fancytitle, right=10pt] at (box.north west) {AC Circuits};
    \end{tikzpicture}

    %------------ Capacitors ---------------
    \begin{tikzpicture}
        \node [mybox] (box){%
            \begin{minipage}{0.3\textwidth}
            % \subtopic{}
            \small{
            \begin{tabular}{lp{8cm} l}
                Capacitance & $C=\dfrac{Q}{V}$\\
                Stored Energy & $U=\dfrac{1}{2}CV^2=\dfrac{1}{2}\dfrac{Q^2}{C}=\dfrac{1}{2}QV$\\
                Capacitors in Series & $\dfrac{1}{C_{eq}}=\dfrac{1}{C_1}+\dfrac{1}{C_2}+\cdots$\\
                Capacitors in Parallel & $C_{eq}=C_1+C_2+\cdots$\\
                Parallel Plate Capacitor & $C=\dfrac{\epsilon_0A}{d}$\\
                Dielectrics & $C_\text{dielectric}=\kappa C_\text{vacuum}$\\
            \end{tabular}
            }
            \end{minipage}
        };
        %------------ Capacitors ---------------------
        \node[fancytitle, right=10pt] at (box.north west) {Capacitors};
        \end{tikzpicture}
    
    
        %------------ Inductors ---------------
       \begin{tikzpicture}
        \node [mybox] (box){%
            \begin{minipage}{0.3\textwidth}
            \small{
            \begin{tabular}{lp{8cm} l}
                Self-Induced EMF & $\mathcal{E}=-L\dfrac{di}{dt}$\\
                Stored Energy & $U=\dfrac{1}{2}LI^2$\\
                Inductors in Series & $L_{eq}=L_1+L_2+\cdots$\\
                Inductors in Parallel & $\dfrac{1}{L_{eq}}=\dfrac{1}{L_1}+\dfrac{1}{L_2}+\cdots$\\
            \end{tabular}
            }
            \subtopic{Solenoids}
            \small{
            \begin{tabular}{lp{8cm} l}
                Coil Density & $n=N/L$\\
                Magnetic Field & $B=\mu_0nI$\\
                Inductance & $L=\dfrac{N\Phi_B}{I}$\\
            \end{tabular}
            }
            \end{minipage}
        };
        %------------ Inductors ---------------------
        \node[fancytitle, right=10pt] at (box.north west) {Inductors};
        \end{tikzpicture}


%------------ Electrostatics ---------------
\begin{tikzpicture}
    \node [mybox] (box){%
        \begin{minipage}{0.3\textwidth}
        \subtopic{Electric Force}
        \small{
        \begin{tabular}{lp{8cm} l}
            Coulomb's Law & $\ds|\vec{F}|=k\frac{|q_1q_2|}{r^2}=q|\vec{E}|$
        \end{tabular}
        }
        \subtopic{Electric Field}
        \small{
        \begin{tabular}{lp{8cm} l}
            Gauss's Law & $\ds\oiint_S\vec{E}\cdot d\vec{A}=\frac{Q_\text{enc}}{\epsilon_0}$\\
            $\vec{E}$ from Point Charge & $\ds\vec{E}=\frac{kq}{r^2}\hat{r}$\\
            $\vec{E}$ from Charged Rod & $\ds E(h)=\frac{kQ}{h\sqrt{h^2+a^2}}$\\
            $\vec{E}$ from Charged Ring & $\ds E(z)=\frac{kQz}{(R^2+h^2)^{3/2}}$\\
            $\vec{E}$ from Charged Disk & $\ds E(z)=\frac{2Qk}{R^2}\left(1-\frac{z}{\sqrt{z^2+R^2}}\right)$\\
            $\vec{E}$ from Infinite Sheet & $\ds E(z)=\frac{\sigma}{2\epsilon_0}\hat{n}$\\
            Flux & $\ds \Phi_E=\iint_S \vec{E}\cdot d\vec{A}$\\
            Energy Density & $u_E=\dfrac{\epsilon_0}{2}E^2$
        \end{tabular}
        }
        \end{minipage}
    };
    %------------ Electrostatics ---------------------
    \node[fancytitle, right=10pt] at (box.north west) {Electrostatics};
    \end{tikzpicture}


    % % --- Authorship ---
    % \begin{center}
    %     \framebox{%
    %         \begin{minipage}[.5cm]{7.5cm} % Adjust the height as needed
    %             \vspace{.2cm}
    %             \centering 
    %             \footnotesize Created January 2024
    %         \end{minipage}
    %     }
    % \end{center}


%------------ Electric Potential ---------------
\begin{tikzpicture}
    \node [mybox] (box){%
        \begin{minipage}{0.3\textwidth}
        \subtopic{Potential}
        \small{
        \begin{tabular}{lp{8cm} l}
            Difference Notation & $V_{ba}=V_b-V_a$\\
            $V$ from Point Charge & $\ds V=\frac{kq}{r}+\text{Constant}$\\
            Potential Difference from $\vec{E}$ & $\ds V=-\int_a^b\vec{E}\cdot d\vec{l}$\\
            Electric Field from $V$ & $\ds\vec{E}=-\nabla V$\\
             & $E_x=-\dfrac{dV}{dx}$
        \end{tabular}
        }
        \subtopic{Potential Energy}
        \small{
        \begin{tabular}{lp{8cm} l}
            Work & $W_{a\to b}=U_a-U_b=-\Delta U$\\
            Potential Energy from $V$ & $U=qV$\\
            Between Point Charges & $U=\dfrac{kq_1q_2}{r}$
        \end{tabular}
        }
        \end{minipage}
    };
    %------------ Electric Potential ---------------------
    \node[fancytitle, right=10pt] at (box.north west) {Electric Potential};
    \end{tikzpicture}


%------------ Magnetostatics ---------------
\begin{tikzpicture}
    \node [mybox] (box){%
        \begin{minipage}{0.3\textwidth}
        \subtopic{Magnetic Force}
        \small{
        \begin{tabular}{lp{8cm} l}
            Lorentz Force & $\vec{F}=q\vec{E}+q\vec{v}\times\vec{B}$\\
            Force on Current & $\vec{F}=I\vec{L}\times\vec{B}$\\
            Force Between Wires & $\ds \frac{F}{L}F=\frac{\mu_0I_1I_2}{2\pi d}$\\
        \end{tabular}
        }
        \subtopic{Magnetic Fields}
        \small{
        \begin{tabular}{lp{8cm} l}
            Biot-Savart Law & $\ds d\vec{B}=\frac{\mu_0}{4\pi}\frac{I\,d\vec{l}\times\hat{r}}{r^2}$\\
            Ampere's Law & $\ds\oint_C\vec{B}\cdot d\vec{l}=\mu_0I_\text{enc}$\\
            Loop of Current & $\ds \vec{B}=\frac{\mu_0IR^2}{2(h^2+R^2)^{3/2}}\hat{n}$\\
            Straight Wire & $\ds B=\frac{\mu_0I}{4\pi r}\sin\theta\bigg\vert_{\theta_L}^{\theta_R}=\frac{\mu_0Ix}{2\pi r\sqrt{x^2+r^2}}\bigg\vert_{x_L}^{x_R}$\\
            Flux & $\ds \Phi_B=\iint_S \vec{B}\cdot d\vec{A}$\\
            Energy Density & $\dfrac{1}{2\mu_0}B^2$
        \end{tabular}
        }
        \subtopic{Torque on Current Loop}
        \small{
        \begin{tabular}{lp{8cm} l}
            Torque Vector & $\vec{\tau}=\vec{\mu}\times\vec{B}$\\
            Magnetic Dipole Moment & $\vec{\mu}=NI\vec{A}$\\
            Potential Energy & $U=-\vec{\mu}\cdot\vec{B}$
        \end{tabular}
        }
        \end{minipage}
    };
    %------------ Magnetostatics ---------------------
    \node[fancytitle, right=10pt] at (box.north west) {Magnetostatics};
    \end{tikzpicture}

   %------------ Maxwell's Equations ---------------
   \begin{tikzpicture}
    \node [mybox] (box){%
        \begin{minipage}{0.3\textwidth}
        % \subtopic{Magnetic Force}
        \small{
        \begin{tabular}{lp{8cm}}
            $\ds\oiint_S\vec{E}\cdot d\vec{A}=\frac{q_\text{enc}}{\epsilon_0}$\\
            $\ds\oiint_S\vec{B}\cdot d\vec{A}=0$\\
            $\ds\oint_C\vec{E}\cdot d\vec{l}=-\frac{d\Phi_B}{dt}$\\
            $\ds\oint_C\vec{B}\cdot d\vec{l}=\mu_0I_\text{enc}+\mu_0\epsilon_0\frac{d\Phi_E}{dt}$
        \end{tabular}
        }
        \end{minipage}
    };
    %------------ Maxwell's Equations ---------------------
    \node[fancytitle, right=10pt] at (box.north west) {Maxwell's Equations};
    \end{tikzpicture}

    %------------ Electromagnetic Induction ---------------
   \begin{tikzpicture}
    \node [mybox] (box){%
        \begin{minipage}{0.3\textwidth}
        % \subtopic{Magnetic Force}
        \small{
        \begin{tabular}{lp{8cm} l}
            Induced EMF & $\ds\mathcal{E} = -\frac{d\Phi_B}{dt}$\\
            Motional EMF & $\ds\mathcal{E}=\oint_C(\vec{v}\times\vec{B})\cdot d\vec{l}$
        \end{tabular}
        }
        \end{minipage}
    };
    %------------ Electromagnetic Induction ---------------------
    \node[fancytitle, right=10pt] at (box.north west) {Electromagnetic Induction};
    \end{tikzpicture}




    %------------ Mechanics ---------------
   \begin{tikzpicture}
    \node [mybox] (box){%
        \begin{minipage}{0.3\textwidth}
        \subtopic{Kinematics}
        \small{
        \begin{tabular}{lp{8cm} l}
            Linear Motion & $x=x_0+\frac{1}{2}(v_0+v)t$\\
             & $x=x_0+vt+\frac{1}{2}at^2$\\
             & $v=v_0+at$\\
             & $v^2=v_0^2+2a(x-x_0)$\\
            Circular Motion & $a_c=\frac{v^2}{r}$
        \end{tabular}
        }
        \subtopic{Forces}
        \small{
        \begin{tabular}{lp{8cm} l}
            Newton's Second Law & $\vec{F}=m\vec{a}=\dfrac{d\vec{p}}{dt}$\\
            % Gravitational Force & $\vec{F}_g=-\frac{GMm}{r^2}\hat{r}$\\
            Spring Force & $\vec{F}=-kx\hat{x}$\\
            Friction Force & $F=\mu N$\\
            Damping Force & $\vec{F}=-b\vec{v}$\\
            Bouyant Force & $\vec{F}=\rho Vg$
        \end{tabular}
        }
        \subtopic{Work and Energy}
        \small{
        \begin{tabular}{lp{8cm} l}
            Work & $\ds W=\int_{\vec{r}_i}^{\vec{r}_f}\vec{F}\cdot d\vec{r}=\vec{F}\cdot\Delta\vec{r}$\\
            Kinetic Energy & $K=\frac{1}{2}mv^2$\\
            Gravitational Potential & $\Delta U_g=mgy$\\
            Spring Potential Energy & $\Delta U_s=\frac{1}{2}kx^2$\\
            Conservative Forces & $\vec{F}=-\nabla U$\\
            Power & $\ds P=\frac{dW}{dt}=\vec{F}\cdot\vec{v}$
        \end{tabular}
        }
        \end{minipage}
    };
    %------------ Mechanics ---------------------
    \node[fancytitle, right=10pt] at (box.north west) {Mechanics};
    \end{tikzpicture}


    %------------ Mathematics ---------------
   \begin{tikzpicture}
    \node [mybox] (box){%
        \begin{minipage}{0.3\textwidth}
        \subtopic{Area and Volume}
        \small{
        \begin{tabular}{lp{8cm} l}
            Volume of a Sphere & $V=\frac{4}{3}\pi r^3$\\
            Volume of a Cylinder & $V=\pi r^2L$\\
            Area of a Sphere & $A=4\pi r^2$\\
            Area of a Cylinder & $A=2\pi rL$\\
            Area of a Circle & $A=\pi r^2$\\
            Circumference of a Circle & $C=2\pi r$
        \end{tabular}
        }
        \subtopic{Trigonometry}
        \small{
        \begin{tabular}{lp{8cm} l}
            Pythagorean Theorem & $a^2+b^2=c^2$\\
            Arc Length & $s=r\theta$\\
            Pythagorean Identity & $\sin^2\theta+\cos^2\theta=1$\\
            Double Angle & $\sin(2\theta)=2\sin\theta\cos\theta$\\
             & $\cos(2\theta)=\cos^2\theta-\sin^2\theta$\\
            Half Angle & $\ds\sin^2(\tfrac{\theta}{2})=\frac{1-\cos\theta}{2}$\\
             & $\ds\cos^2(\tfrac{\theta}{2})=\frac{1+\cos\theta}{2}$\\
        \end{tabular}
        }
        \subtopic{Integrals}
        \small{
        \begin{tabular}{lp{8cm}}
            $\ds\int x^ndx=\begin{cases}
                \frac{x^{n+1}}{n+1}+\text{Constant} & n\neq -1\\
                \ln |x|+\text{Constant} & n=-1
            \end{cases}$
        \end{tabular}\\
        }
        \subtopic{Vectors}
        \small{
        \begin{tabular}{lp{8cm} l}
            Dot Product & $\vec{a}\cdot\vec{b}=ab\cos\theta$\\
            Cross Product & $\|\vec{a}\times\vec{b}\|=ab\sin\theta$\\
             & $\vec{a}\times\vec{b}=\left|\begin{matrix}\hat{i} & \hat{j} & \hat{k}\\ a_x & a_y & a_z\\ b_x & b_y & b_z\end{matrix}\right|$\\
        \end{tabular}
        }
        \subtopic{Right Hand Rule}
        \centerline{\includegraphics[scale=0.5]{right_hand_rule.png}}
        \end{minipage}
    };
    %------------ Mathematics ---------------------
    \node[fancytitle, right=10pt] at (box.north west) {Mathematics};
    \end{tikzpicture}

\end{multicols*}

\end{document}