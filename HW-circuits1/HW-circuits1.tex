\documentclass[11pt, fleqn]{article}
\usepackage[utf8]{inputenc}
\usepackage{fullpage}
\usepackage{amsmath,amssymb,array}
\usepackage{dsfont}
\usepackage{amsfonts}
\usepackage{graphicx}
\usepackage{mathtools}
\usepackage{polynom}
\usepackage{esint}
\usepackage{mathrsfs}
\usepackage{fizztex}
\usepackage{amsthm}
\setlength{\parindent}{0em}





\title{Physics 158 Circuits Homework 1}
\author{}
\date{}

\begin{document}
\allowdisplaybreaks

\maketitle

Aim for this homework is to cover the following topics:
\begin{itemize}
    \item Introductory resistor circuits problems that can be used as a review for students who have not seen them in high school (aim for 1 problem)
    \item Analyzing circuits with capacitors, inductors, and switches in initial states and after long periods of time (aim for 2-3 problems)
    \item Analyzing RC and RL circuits and time constants (aim for 2-3 problems)
\end{itemize}

Review topics/guides to support these topics:
\begin{itemize}
    \item Can include a document and optional MP homework assignment that introduces students to the concept of circuits if they haven't seen them before
    \item Can have a short guide or example going over how to find the expression for current/voltage of a time dependent RC/LR circuit using ODEs. Students are not expected to know this but it can be very helpful for them to know so we should provide an optional introduction to the topic for those interested in the modivation behind some of these equations.
\end{itemize}

% circuits starts week 4

\section*{Learning Goals to Address}
\begin{enumerate}
    \item Describe how series and parallel circuits function and how they differ 
    \item Apply Kirchhoff’s laws to various simple and complex series and parallel circuits
    \item Apply Ohm's law to various simple and complex series and parallel circuits
    \item Understand how to apply and calculate voltage, resistance and current rules to various simple and complex series and parallel circuits 
    \item Distinguish between EMF, potential difference and terminal voltage and determine their relation to the internal resistance of a battery 
    \item Understand and be able to calculate power and energy within various circuits, including energy transfer and power dissipation 
    \item Distinguish between capacitors and inductors 
    \item Calculate the equivalent capacitance and inductance of various circuits 
    \item Calculate the energy stored in a capacitor and in an inductor 
    \item Calculate the time dependence of current charge and potential difference for a charging and discharging capacitor in an RC circuit (graphically as well)
    \item Determine the time dependence of current charge and potential difference for a charging and discharging inductor in an LR circuit (graphically as well)
    \item Understand the relevance and representation of the time constant in RC and LR circuits
    \item Calculate the time constant within various circuits, including RC and LR circuits
\end{enumerate}
Notes on the learning goals:
\begin{itemize}
    \item Learning goals 1-4 will be addressed in most, if not all, problems
    \item I never remember covering learning goal 5. Can include that in problem 1 if necessary
    \item Learning goal 9 can likely be better addressed in the 2nd circuits homework set with RLC circuits
    \item 
\end{itemize}



\section*{Brainstorming}
Tutorial 3 problem 2 is very good for expressing conservation of charge and equivalent capacitance (same with HW3Q1)\\
Thought for tutorials: create simple problems that encourage design (this is for engineers after all). One example would be design a circuit that turns on a light for x seconds when the battery is disconnected.


\section*{Problem 1 (Resistor Circuits)}

\section*{Problem 2 (Capacitor Conservation of Charge)}
Tutorial 3 Q2\\
HW 3 Q1 (clarify the circuit diagram. A lot of questions on this)

\section*{Problem 3 (Time Dependent Circuit With Switches)}
HW 3 Q2\\
HW 3 Q4\\
HW 3 Q5 (will reword slightly to clarify the arrows)

\section*{Problem 4 (RC/LR Time Dependent)}
HW 3 Q3

\section*{Problem 5 (Application of RC/LR Circuits)}
\textit{Learning goals 6,9,11,12,13}\\
Using their newfound knowledge of LR circuits, a Phys 158 student came up with a clever idea for a prank. They want to design an alarm clock that will continue to play for 10 seconds after the battery is removed. The alarm clock can be thought of as a $1\units{k\Omega}$ resistor which requires at least 1 Watt to operate. They designed the following circuit to achieve this.\\
\includegraphics{Images/AlarmClockCircuit}\\
a) What value should the battery be such that the power supplied to the clock does not exceed 3 Watts?\\
b) What value of inductor should they use so that the alarm clock remains on for 10 seconds after the battery is disconnected?\\
c) Plot the energy stored in the inductor as a function of time after the batter has been disconnected.\\

\textbf{Solution:}\\
a)
\begin{align*}
    &P_C=3\units{W}\\
    &P_C=i_C^2R_C\Ra i_C=\sqrt{\frac{P_C}{R_C}}\\
    &\varepsilon=V_C=i_CR_C=\sqrt{\frac{P_C}{R_C}}\cdot R_C=\sqrt{P_CR_C}=\sqrt{(3\units{W})(1000\units{\Omega})}=54.8\units{V}
\end{align*}
b)\\
We can start by writing out Kirchoff's loop voltage law for the circuit and then solving the resulting ODE to get an expression for the current as a function of time:
\begin{align*}
    &iR+iR_C+L\ddx[i]{t}=0\\
    &L\ddx[i]{t}=-i(R+R_C)\\
    &\ddx[i]{t}=-\frac{R+R_C}{L}i\\
    &\frac{di}{i}=-\frac{R+R_C}{L}dt\\
    &\int\frac{di}{i}=-\frac{R+R_C}{L}\int dt\\
    &\ln|i|=-\frac{R+R_C}{L}t+\text{Constant}\\
    &i(t)=e^{-\frac{R+R_C}{L}t+\text{Constant}}=i_0e^{-\frac{R+R_C}{L}t}
\end{align*}
Alternatively, we can get the same expression by thinking about it conceptually and computing the time constant.\\
We know that the current will initially want to stay the same because of the inductor and will slowly decay to 0 so we can determine that the equation of the current should look like exponential decay and be of the form
\[i(t)=i_0e^{-t/\tau}\]
We can then compute the time constant for an RL circuit as
\[\tau=\frac{L}{R_{eq}}=\frac{L}{R+R_C}\]
Plugging this in will yield the same expression as above.\\
The initial current, $i_0$, will be the current that was initially flowing through the inductor. We computed this in part (a) to be
\[i_0=i_C=\sqrt{\frac{P_C}{R_C}}=\sqrt{\frac{3\units{W}}{1000\units{\Omega}}}=54.8\units{mA}\]
Now we have a complete expression for the current as a function of time. We can get the power as a function of time as
\begin{align*}
    &P(t)=i^2R_C=i_0^2R_Ce^{-\frac{2(R+R_C)}{L}t}\\
    &i_0^2R_C=\frac{P_C}{R_C}\cdot R_C=P_C\\
    &P(t)=P_Ce^{-\frac{2(R+R_C)}{L}t}
\end{align*}
We are told that the clock must have a minimum of 1 Watt and we want it to last for 10 seconds so we can set $P=1$ and $t=10$ and solve for $L$.
\begin{align*}
    &\frac{P}{P_C}=e^{-\frac{2(R+R_C)}{L}t}\\
    &\ln\bfrac{P}{P_C}=-\frac{2(R+R_C)}{L}t\\
    &L=-\frac{2(R+R_C)t}{\ln\bfrac{P}{P_C}}=-\frac{2(100\units{\Omega}+1000\units{\Omega})(10\units{s})}{\ln\bfrac{1\units{W}}{3\units{W}}}=20,025\units{H}
\end{align*}












\end{document}
