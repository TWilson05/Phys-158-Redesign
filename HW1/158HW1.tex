\documentclass[11pt, fleqn]{article}
\usepackage[utf8]{inputenc}
\usepackage{fullpage}
\usepackage{amsmath,amssymb,array}
\usepackage{dsfont}
\usepackage{amsfonts}
\usepackage{graphicx}
\usepackage{mathtools}
\usepackage{polynom}
\usepackage{esint}
\usepackage{mathrsfs}
\usepackage{fizztex}
\usepackage{amsthm}
\setlength{\parindent}{0em}





\title{Physics 158 Waves and Interference Homework}
\author{}
\date{}

\begin{document}
\allowdisplaybreaks

\maketitle

\section*{Thoughts on Problems}
\begin{enumerate}
    \item In-depth double slit experiment problem (hard)
    \item interference problem (one easy, one hard)
    \item Beats problem? (easy)
    \item Standing wave problem
    \item Thin film (easy)
\end{enumerate}

\section*{Problem 1}
\textit{Difficulty: 2/5}\\
\textit{Topics: interference, phase shift}\\
A speaker sits locked at the center of a spherical room. Another speaker
of the same frequency but out of phase by $\frac{\pi}{3}$ can be moved around. 
What locations could you move this other speaker so that constructive interference is
heard at the centre of the room?

\section*{Problem 2}
\textit{Difficulty: 2/5}\\
\textit{Topics: interference}\\
\textit{Source: Phys 158 Tutorial 2}
A sound wave of $40.0\units{cm}$ wavelength enters a tube as shown. What is the
smallest radius, $r$, such that a minimum is heard by the detector?\\
\centerline{\includegraphics{Images/InterferenceTube.png}}

\textbf{Solution}\\
There are two possible paths that the waves can take.
Either the top path through the circular section, or the bottom path through the straight section.\\
The length of the upper path is
\begin{align*}
    &l_1+l_2+\pi r
\end{align*}
The length of the lower path is
\begin{align*}
    &l_1+l_2+2r
\end{align*}
So the path difference is then
\begin{align*}
    &\Delta r=\pi r-2r
\end{align*}
For a minimum to be heard, we require destructive interference which requires $\Delta r$ is a half wavelength difference.
\begin{align*}
    &\Delta r=\brround{n+\frac{1}{2}}\lambda,\ n\in \Z\\
    &(\pi -2)r=\brround{n+\frac{1}{2}}\lambda\\
    &r=\frac{\brround{n+\frac{1}{2}}\lambda}{\pi-2}
\end{align*}
We then want to find the smallest possible radius which corresponds to plugging in $n=0$.
\begin{align*}
    &r=\frac{\frac{\lambda}{2}}{\pi-2}=\frac{\frac{1}{2}(40\units{cm})}{\pi-2}=17.5\units{cm}
\end{align*}

\section*{Problem 3}
\textit{Difficulty: 2/5}\\
\textit{Topics: interference}\\
\textit{Source: Phys 158 Tutorial 2}\\
Two loudspeakers are placed $4.0\units{m}$ apart on a stage. The speakers generate
identical sound waves when tested in a range of $20\units{Hz}$ to $20,000\units{Hz}$. A person sits 
directly in front of one speaker at a distance of $10.0\units{m}$. Assume that the velocity of
sound is $344\units{m/s}$.\\
a) What is the lowest frequency for which the person will hear a maximum intensity?\\
b) What are the two highest frequencies for which the person will hear a minimum intensity?\\

\textbf{Solution}

\section*{Problem 4}
\textit{Source: Phys 158 Homework 1}\\
A Physics 158 student observes that a pipe produces resonances at successive frequencies of 629, 743, and 856 Hz. Analyze this data to determine whether the pipe is a STOPPED pipe (ie open ONLY at 1 end)
or an OPEN pipe (ie open at both ends). Determine the fundamental frequency and length of this pipe.\\
(Use $v_\text{sound} = 343 \units{m/s}$).

\section*{Problem 5}
\textit{Source: Phys 158 Homework 1}\\
An aluminum wire of length $l_1 = 60.0 \units{cm}$ with a cross-sectional area $A = 1.00 x 10^{-2} \units{cm^2}$ 
and density of $2.60 \units{g/cm^3}$ is connected to a steel wire of density $7.80 \units{g/cm^3}$ with the same cross-sectional area. The
compound wire is loaded with a block of mass $m=10.0 \units{kg}$ and is arranged as shown below such that the
distance $l_2$ from the joint to the supporting pulley is $86.60 \units{cm}$. Transverse waves are set up in the wire by
an external source of variable frequency.\\
a) Find the lowest frequency of excitation for which standing
waves are observed such that the joint in the wire is a NODE.\\
b) Sketch the transverse displacement of the
wires for this lowest harmonic.\\
c) How many nodes are observed at this frequency?\\
(NOTE: the frequency of the standing waves is the same in both wires).\\
\centerline{\includegraphics{Images/CompoundWire.png}}

\section*{Problem 6}
\textit{Source: Phys 158 Homework 1}\\
Two speakers $S_1$ and $S_2$ are located $4\units{m}$ apart as shown in the picture below. They are exactly in
phase and emit waves with a frequency = 600 Hz. A Physics 158 student walks along a line perpendicular to
the line joining the 2 speakers as shown. What is the \underline{smallest y value} at which she will hear \underline{Minimum} sound?\\
\centerline{\includegraphics{Images/RightAngleInterference.png}}

\section*{Problem 7}
\textit{Source: Phys 158 Homework 1}\\
Suppose that there is a phase shift of $\frac{\pi}{3}$ radians between the 2 speakers (ie $\phi_1-\phi_2=\frac{\pi}{3}$) due to the
different lengths of wire connecting them to the same amplifier. What is the smallest y value at which our
Physics 158 student will now hear Minimum sound ?

\section*{Problem 8}
\textit{Source: Phys 158 Homework 1}\\
Consider the interference of 2 sound waves emitted from speakers that are separated by a distance $d =
1\units{m}$ along a line that is located a distance $R = 2.5\units{m}$ as shown in Activity 2 of the posted class notes. Using
$v_\text{sound} = 343 \units{m/s}$ and $f = 1500 \units{Hz}$ use a spreadsheet (MS Excel or Google spreadsheets) or the DESMOS
plotting program to determine the 4 lowest MINIMUM y values numerically -- to an accuracy of 5\%. Please
include a copy of your spreadsheet values or graph along with your 4 $y_\text{min}$ values. You can find information
on how to use a spreadsheet to rapidly calculate $\Delta r_{12}$ for several $d_n$ values at the link below --\\
https://multimedia.journalism.berkeley.edu/tutorials/spreadsheets/\\
a) Method 1 -- calculate $(r_2-r_1-m\cdot\frac{\lambda}{2})$ for all values of $y_m$ using a spreadsheet
enter $y_m$ values in column $A_i$\\
calculate $r_1 = \sqrt{(y_m-\frac{d}{2})^2+R^2}$ and $r_2$ as shown in Activity 2, and then
determine the values of $y_m$ that make the above quantity $(r_2-r_1-m\cdot\frac{\lambda}{2})=0$.\\
b) Method 2 -- use a plotting program such as DESMOS to plot $r_2-r_1$ versus $y_m$ and determine the values of $y_m$ which make $\Delta r=m\cdot\frac{\lambda}{2}$

\section*{Problem 9}
\textit{Source: Phys 158 Sample Midterm 1A}\\
Given two speakers as shown below both producing sound waves ($v_\text{sound} =
343\units{m/s}$) at $f = 400\units{Hz}$, $L = 2\units{m}$, initially one half a wavelength out of phase with each
other. Find the smallest $y$ so that a maximum is heard at the point a. Note: the lines in
the diagram below form a right triangle.\\
\centerline{\includegraphics{Images/RightAngleInterference2.png}}

\section*{Problem 10}
\textit{Source: Phys 158 Sample Midterm 1B}\\
An omni-directional microwave source (waves goes in all directions) with
$\lambda= 0.86\units{m}$ is placed next to a mirror as shown in the diagram below with $L = 1\units{m}$.\\
a) There are spots of maximum intensity and minimum intensity along the line between
P and the source. Find the smallest value of $y \neq 0$ for which a minimum intensity is detected.\\
b) A detector located a large fixed distance ($y \gg 0$) from the source is now moved
clockwise along a semi-circular path centred at the source. Find how many maximum
intensity spots will be observed by this detector as it moves from the positive y-axis to the
negative y-axis.\\
\centerline{\includegraphics{Images/CircularLight.png}}

\section*{Problem 11}
\textit{Source: Phys 158 Sample Midterm 1C}\\
An optical filter is a device which preferentially transmits desired wavelengths of light while reflecting others.
A simple optical filter can be constructed from a thin film of titanium dioxide ($n_f = 2.68$) deposited on optical
glass ($n_g = 1.53$), as shown in the diagram. You may treat the layer of glass as being much thicker than the layer
of $\mathrm{TiO_2}$ and assume normal incidence for the light rays. The spectrum of visible light ranges from 390–700 nm.\\
(a) If three of the wavelengths of electromagnetic radiation maximally reflected by the filter are $\lambda_1 = 2070 \units{nm}$
(infrared),  $\lambda_2 = 690 \units{nm}$ (red light), and  $\lambda_3 = 414 \units{nm}$ (violet light), what is the thickness of the thin film?\\
(b) What wavelength of visible light is preferentially transmitted by this optical filter? (Hint: if an incident wave
is preferentially transmitted, what must be true of its reflected waves?) If you are unable to determine the
actual film thickness in part (a) you can use $t = 250\units{nm}$. NOTE–THIS IS NOT THE CORRECT ANSWER.\\
\centerline{\includegraphics{Images/ThinFilmBlock.png}}







\end{document}
