\documentclass[11pt, fleqn]{article}
\usepackage[utf8]{inputenc}
\usepackage{fullpage}
\usepackage{amsmath,amssymb,array}
\usepackage{dsfont}
\usepackage{amsfonts}
\usepackage{graphicx}
\usepackage{mathtools}
\usepackage{polynom}
\usepackage{esint}
\usepackage{mathrsfs}
\usepackage{fizztex}
\usepackage{amsthm}
\setlength{\parindent}{0em}





\title{Physics 158 Waves and Interference Homework}
\author{}
\date{}

\begin{document}
\allowdisplaybreaks

\maketitle

\section*{Thoughts on Problems}
\begin{enumerate}
    \item In-depth double slit experiment problem (hard)
    \item interference problem (one easy, one hard)
    \item Beats problem? (easy)
    \item Standing wave problem
    \item Thin film (easy)
\end{enumerate}

\section*{Problem 1}
\textit{Difficulty: 2/5}\\
\textit{Topics: interference, phase shift}\\
A speaker sits locked at the center of a spherical room. Another speaker
of the same frequency but out of phase by $\frac{\pi}{3}$ can be moved around. 
What locations could you move this other speaker so that constructive interference is
heard at the centre of the room?

\section*{Problem 2}
\textit{Difficulty: 2/5}\\
\textit{Topics: interference}\\
\textit{Source: Phys 158 Tutorial 2}
A sound wave of $40.0\units{cm}$ wavelength enters a tube as shown. What is the
smallest radius, $r$, such that a minimum is heard by the detector?\\
\centerline{\includegraphics{Images/InterferenceTube.png}}

\section*{Problem 3}









\end{document}
