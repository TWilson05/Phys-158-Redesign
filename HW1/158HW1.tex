\documentclass[11pt, fleqn]{article}
\usepackage[utf8]{inputenc}
\usepackage{fullpage}
\usepackage{amsmath,amssymb,array}
\usepackage{dsfont}
\usepackage{amsfonts}
\usepackage{graphicx}
\usepackage{mathtools}
\usepackage{polynom}
\usepackage{esint}
\usepackage{mathrsfs}
\usepackage{fizztex}
\usepackage{amsthm}
\setlength{\parindent}{0em}





\title{Physics 158 Waves and Interference Homework}
\author{}
\date{}

\begin{document}
\allowdisplaybreaks

\maketitle

\section*{Thoughts on Problems}
\begin{enumerate}
    \item In-depth double slit experiment problem (hard)
    \item interference problem (one easy, one hard)
    \item Beats problem? (easy)
    \item Standing wave problem
    \item Thin film (easy)
\end{enumerate}

\section*{Problem 1}
\textit{Difficulty: 2/5}\\
\textit{Topics: interference, phase shift}\\
A speaker sits locked at the center of a spherical room. Another speaker
of the same frequency but out of phase by $\frac{\pi}{3}$ can be moved around. 
What locations could you move this other speaker so that constructive interference is
heard at the centre of the room?

\section*{Problem 2}
\textit{Difficulty: 2/5}\\
\textit{Topics: interference}\\
\textit{Source: Phys 158 Tutorial 2}
A sound wave of $40.0\units{cm}$ wavelength enters a tube as shown. What is the
smallest radius, $r$, such that a minimum is heard by the detector?\\
\centerline{\includegraphics{Images/InterferenceTube.png}}

\textbf{Solution}\\
There are two possible paths that the waves can take.
Either the top path through the circular section, or the bottom path through the straight section.\\
The length of the upper path is
\begin{align*}
    &l_1+l_2+\pi r
\end{align*}
The length of the lower path is
\begin{align*}
    &l_1+l_2+2r
\end{align*}
So the path difference is then
\begin{align*}
    &\Delta r=\pi r-2r
\end{align*}
For a minimum to be heard, we require destructive interference which requires $\Delta r$ is a half wavelength difference.
\begin{align*}
    &\Delta r=\brround{n+\frac{1}{2}}\lambda,\ n\in \Z\\
    &(\pi -2)r=\brround{n+\frac{1}{2}}\lambda\\
    &r=\frac{\brround{n+\frac{1}{2}}\lambda}{\pi-2}
\end{align*}
We then want to find the smallest possible radius which corresponds to plugging in $n=0$.
\begin{align*}
    &r=\frac{\frac{\lambda}{2}}{\pi-2}=\frac{\frac{1}{2}(40\units{cm})}{\pi-2}=17.5\units{cm}
\end{align*}

\section*{Problem 3}
\textit{Difficulty: 2/5}\\
\textit{Topics: interference}\\
\textit{Source: Phys 158 Tutorial 2}\\
Two loudspeakers are placed $4.0\units{m}$ apart on a stage. The speakers generate
identical sound waves when tested in a range of $20\units{Hz}$ to $20,000\units{Hz}$. A person sits 
directly in front of one speaker at a distance of $10.0\units{m}$. Assume that the velocity of
sound is $344\units{m/s}$.\\
a) What is the lowest frequency for which the person will hear a maximum intensity?\\
b) What are the two highest frequencies for which the person will hear a minimum intensity?\\

\textbf{Solution}











\end{document}
