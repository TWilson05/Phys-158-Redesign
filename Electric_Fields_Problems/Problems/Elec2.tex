\subsection*{Problem 2}
\iftoggle{contributions}{
\textit{Created by Tyler Wilson 2023}
}{}

An insulating cylinder of radius $R=5\units{cm}$ and effectively infinite length in the z-direction contains a uniform charge density of $10\units{C/m^3}$.
\begin{enumerate}
    \item Find the electric field everywhere in space
\end{enumerate}
If there is now a \underline{hollow} spherical cavity of radius $a=1\units{cm}$ located at the center of the cylinder,
\centerline{\includegraphics[width=\textwidth]{/Images/P2img1.png}}

\begin{enumerate}
    \setcounter{enumi}{1}
    \item Find the electric field at the center of the sphere at point A.
    \item Find the electric field just outside the sphere at point B.
    \item Find the electric field just outside the sphere at point C.
    \item If $h=10\units{cm}$, find the electric field outside both objects at point D.
\end{enumerate}
If the potential at the point X is 0V,
\begin{enumerate}
    \setcounter{enumi}{5}
    \item What is the potential at point A?
\end{enumerate}

\iftoggle{solutions}{
\textbf{Solutions:}\\
\begin{enumerate}
\item We can use Gauss's Law to find the electric field everywhere in space. We will use a Gaussian cylinder of radius $r$ and length $L$.
\begin{align*}
    &\oiint\vec{E}\cdot d\vec{A}=\frac{Q_\text{enc}}{\epsilon_0}
\end{align*}
We will need to break this into two parts: the inside of the cylinder and the outside of the cylinder.\\
For the inside of the cylinder,
\begin{align*}
    &\oiint\vec{E}_\text{in}\cdot d\vec{A}=E_\text{in}A=\frac{Q_\text{enc}}{\epsilon_0}\\
    &Q_\text{enc}=\rho V_\text{enc}=\rho\pi r^2L\\
    &E_\text{in}A=\frac{\rho\pi r^2L}{\epsilon_0}\\
    &2\pi rLE_\text{in}=\frac{\rho\pi r^2L}{\epsilon_0}\\
    &E_\text{in}=\frac{\rho r}{2\epsilon_0}\\
    &\answer{\vec{E}_\text{in}=\frac{\rho r}{2\epsilon_0}\hat{r}}
\end{align*}
For the outside of the cylinder,
\begin{align*}
    &\oiint\vec{E}_\text{out}\cdot d\vec{A}=E_\text{out}A=\frac{Q_\text{enc}}{\epsilon_0}\\
    &Q_\text{enc}=\rho V_\text{enc}=\rho\pi R^2L\\
    &E_\text{out}A=\frac{\rho\pi R^2L}{\epsilon_0}\\
    &2\pi rLE_\text{out}=\frac{\rho\pi R^2L}{\epsilon_0}\\
    &E_\text{out}=\frac{\rho R^2}{2\epsilon_0r}\\
    &\answer{\vec{E}_\text{out}=\frac{\rho R^2}{2\epsilon_0r}\hat{r}}
\end{align*}
\item 
\end{enumerate}
}{}