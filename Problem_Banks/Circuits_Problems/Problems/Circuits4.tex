\iftoggle{title}{
\subsection*{Problem 4}
}{}
\iftoggle{contributions}{
\textit{2023 PHYS 158 Tutorial 4 Question 2}
}{}

\iftoggle{difficulty}{
Difficulty: $\medblackstar \medblackstar \medwhitestar$
}{}

The circuit below has the switch S is opened for a long time. $R_1=2\units{\Omega},\ R_2=4\units{\Omega},\ C=2\units{F}$\\
\centerline{\includegraphics[width=0.6\textwidth]{/Images/P4img1.png}}
\begin{enumerate}
    \item The switch S is now closed. Find all currents just after the switch is closed.
    \item Find all currents after the switch has been closed for a very long time.
    \item After the switch was closed for a very long time it is opened again find the current
    through $R_2$ as a function of time.
\end{enumerate}

\iftoggle{solutions}{
\textbf{Solution:}
\begin{enumerate}
    \item The capacitor will want to initially act as a wire so we can analyze the circuit as two resistors in parallel. Due to Kirchoff's loop law, we can say that each resistor must have a voltage drop of $12\units{V}$ and we can get the current of each from Ohm's law:
    \begin{align*}
        &I_1=\frac{\varepsilon}{R_1}=\frac{12}{2}=\answer{6\units{A}}\\
        &I_2=\frac{\varepsilon}{R_2}=\frac{12}{4}=\answer{3\units{A}}
    \end{align*}
    \item After the switch has been closed for a long time, the capacitor will be fully charged and act as a short circuit. The circuit can then be analyzed as the loop going through the battery and $R_1$
    \begin{align*}
        &I_1=\frac{\varepsilon}{R_1}=\frac{12}{2}=\answer{6\units{A}}\\
        &\answer{I_2=0\units{A}}
    \end{align*}
    \item After the switch is opened the current will flow through the loop containing $R_1$, $R_2$, and $C$. We can write the voltage loop equation as
    \[0=V_C+V_{R_1}+V_{R_2}\]
    \[0=\frac{q}{C}+iR_1+iR_2\]
    We know that $i=\ddx[q]{t}$ and can take the derivative of both sides to get a 1st order differential equation and solve for $i(t)$
    \begin{align*}
        &0=\frac{i}{C}+\ddx[i]{t}(R_1+R_2)\\
        &\ddx[i]{t}=-\frac{i}{(R_1+R_2)C}\\
        &\frac{di}{i}=-\frac{dt}{(R_1+R_2)C}\\
        &\int\frac{di}{i}=-\int\frac{dt}{(R_1+R_2)C}\\
        &\ln|i|=-\frac{t}{(R_1+R_2)C}+\text{Constant}\\
        &i=i_0e^{-\frac{t}{(R_1+R_2)C}}
    \end{align*}
    We can solve for the initial current by using our same voltage loop equation and knowing that the initial voltage across the capacitor is $12\units{V}$ from the charge stored on it. The capacitor will be discharging so the potential in the equation can be thought of as negative.
    \begin{align*}
        &0=V_C+i_0(R_1+R_2)\\
        &0=-12+6i_0\Ra i_0=2\units{A}
    \end{align*}
    Plugging this all in we get,
    \begin{align*}
        &\answer{i(t)=2e^{-\frac{t}{12}}\text{ Amps}}
    \end{align*}
\end{enumerate}
}{}