\iftoggle{title}{
\subsection*{Problem 16}
}{}
\iftoggle{contributions}{
\textit{Created by Tyler Wilson 2023}
}{}

\iftoggle{difficulty}{
Difficulty: $\medblackstar \medwhitestar \medwhitestar$
}{}   If the switch is open,  find all currents and potentials at the labelled points.\\

If the switch is then closed,  find all currents and potentials at the labelled points.\\

$E=12\units{V}$, $R_1=7\units{\Omega}$, $R_2=4\units{\Omega}$, $R_3=10\units{\Omega}$
\subimport*{/Images}{P16img1.tex}

\iftoggle{solutions}{
\textbf{Solution:}
\begin{enumerate}
    \item When the switch is open, there is a short circuit in the branch with $R_3$ so the circuit will act as a series circuit containing $R_1$, $R_2$, and $E$. The potential and voltage at point $c$ will be 0.\\
    The equivalent resistance will be $R_\text{eq}=R_1+R_2$. The current can be computed from Ohm's law:
    \begin{align*}
        &E=I_aR_\text{eq}\\
        &I_a=I_b=\frac{E}{R_\text{eq}}=\frac{E}{R_1+R_2}=\frac{12\units{V}}{11\units{\Omega}}=\answer{1.09\units{A}}
    \end{align*}
    Point $b$ is connected to ground so the potential of point $b$ will be $0\units{V}$. The potential of point $a$ can be computed as the voltage drop across $R_2$.
    \begin{align*}
        &V_{R_2}=V_{ab}=V_a-V_b=V_a\\
        &V_{R_2}=I_aR_2=\frac{12}{11}\units{A}\cdot 4\units{\Omega}=\answer{4.36\units{V}}
    \end{align*}
    \item When the switch is closed, $R_2$ and $R_3$ will be in parallel. We can compute the equivalent resistance for the circuit as
    \begin{align*}
        &\frac{1}{R_{23}}=\frac{1}{R_2}+\frac{1}{R_3}=\frac{1}{4}+\frac{1}{10}=\frac{7}{20}\Ra R_{23}=\frac{20}{7}\units{\Omega}=2.67\units{\Omega}\\
        &R_\text{eq}=R_1+R_{23}=7+\frac{20}{7}=\frac{69}{7}\units{\Omega}
    \end{align*}
    Then we can compute the total current as
    \begin{align*}
        &E=I_aR_\text{eq}\Ra I_a=\frac{E}{R_\text{eq}}=\frac{12\units{V}}{\frac{69}{7}\units{\Omega}}=\answer{1.22\units{A}}
    \end{align*}
    We can use this to compute the volatge across $R_1$
    \begin{align*}
        &V_{R_1}=I_aR_1=8.52\units{V}
    \end{align*}
    Then we can use Kirchoff's voltage law to find the remaining voltages
    \begin{align*}
        &V_{R_2}=V_{R_3}\\
        &V_E=V_{R_1}+V_{R_2}\Ra V_{R_2}=V_{R_3}=V_E-V_{R_1}=3.48\units{V}
    \end{align*}
    From this we can get that the potential at $c$ and the potential at $a$ will be $V_a=V_c=\answer{3.48\units{V}}$ while the potential at $b$ will be $0\units{V}$ as in part a.\\
    We can also use the voltages across the resistors to compute the currents in each branch.
    \begin{align*}
        &I_b=\frac{V_{R_2}}{R_2}=\frac{3.48}{4}=\answer{0.87\units{A}}\\
        &I_c=\frac{V_{R_3}}{R_3}=\frac{3.48}{10}=\answer{0.35\units{A}}
    \end{align*}
\end{enumerate}
}{}
