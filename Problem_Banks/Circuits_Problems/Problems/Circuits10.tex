\iftoggle{title}{
\subsection*{Problem 10}
}{}
\iftoggle{contributions}{
\textit{2023 PHYS 158 Homework 3 Question 4}
}{}

\iftoggle{difficulty}{
Difficulty: $\medblackstar \medblackstar \medwhitestar$
}{}

The circuit below has been open for a very long time and then the switch is closed at $t=0$.\\
\centerline{\includegraphics[width=0.5\textwidth]{Images/P10img1.png}}
\begin{enumerate}
    \item Find all of the currents at $t=0^+$.
    \item Find all of the currents after a very long time.
    \item Find all of the currents as a function of time
\end{enumerate}

\iftoggle{solutions}{
\textbf{Solution:}\\
\includegraphics[width=0.5\linewidth]{Images/P10img2.png}
\vspace{5mm}

\begin{enumerate}
	\item Close switch at $t=0$, then
	$I_2(0)=0$ and the
	Inductor acts like a Battery
	
	$I_0(0)=I_1(0)$
	$$\mathcal{E} - I_1R_1=0 \implies \underline{I_1=\frac{\mathcal{E}}{R_1}=I_0}$$
	
	\item As $t\rightarrow \infty$ the Inductor voltage $\rightarrow 0$ (becomes a wire)
	
	\vspace{5mm}
	\includegraphics[width=0.4\linewidth]{Images/P10img3.png}
	\vspace{5mm}
	
	Now $I_0' = I_1' +I_2'$:
	
	$$I_1'R_1=I_2'R_2=\mathcal{E}=I$$
	
	$$\therefore \underline{I_1'=\frac{\mathcal{E}}{R_1}, \: I_2'=\frac{\mathcal{E}}{R_2}, \: I_0'=\mathcal{E}\left(\frac{1}{R_1}+\frac{1}{R_2}\right)}$$
	
	
	\item At any time $t$ we have
	
	$K_1 \implies I_0(t) = I_1(t)+I_2(t)$
	
	$K_2 \implies \begin{cases}
		\mathcal{E} - I_1R_1 = 0\\
		\mathcal{E} - I_2R_2 -L\frac{dI_2}{dt}=0
	\end{cases} $
	\vspace{5mm}
	
	This last equation looks like a regular RL circuit, so its solution is $I_2(t)=A\exp(-t/\tau)+B$ with $\tau=$ time constant
	
	$$I_2(\infty)\rightarrow \frac{\mathcal{E}}{R_2}=B, \: I_2(0)=A+B=0 \implies A=\frac{\mathcal{E}}{R_2}$$
	
	so $I_2(t)=-\frac{\mathcal{E}}{R_2}\exp(-t/\tau)+\frac{\mathcal{E}}{R_2}=\frac{\mathcal{E}}{R_2}\left(1-\exp(t/\tau\right)$.
	\vspace{5mm}
	
	To find $\tau$ we use the aforementioned equation that looks like a regular RL circuit
	$$\mathcal{E}-R_2\left(\frac{\mathcal{E}}{R_2}\left(1-\exp(-t/\tau\right)\right)-\frac{L\mathcal{E}}{R_2}\left(\frac{1}{\tau}\exp(-t/\tau)\right)=0.$$
	
	$$\therefore \mathcal{E} - \mathcal{E} + \mathcal{E}\exp(-t/\tau)-\frac{L\mathcal{E}}{R_2}\frac{1}{\tau}\exp(-t/\tau)=0,$$
	
	and all $\mathcal{E}$ can be divide out, resulting in
	$$\left(1-\frac{L}{R_2}\frac{1}{\tau}\right)=0 \implies \tau = \frac{L}{R_2}$$
	
	$$\implies \underline{I_2= \frac{\mathcal{E}}{R_2}(1-\exp(-tR_2/\tau))}$$
	
	$$\underline{I_1=\frac{\mathcal{E}}{R_1} \:\: I_0=\frac{\mathcal{E}}{R_1}+\frac{\mathcal{E}}{R_2}\left(1-\exp(-tR_2/L)\right)}$$
\end{enumerate}
}{}