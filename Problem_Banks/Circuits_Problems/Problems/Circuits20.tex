\iftoggle{title}{
\subsection*{Problem 20}
}{}
\iftoggle{contributions}{
\textit{Created by Tyler Wilson 2023}
}{}

\iftoggle{difficulty}{
Difficulty: $\medblackstar \medwhitestar \medwhitestar$
}{}

The LRC circuit below is constructed with a $12\units{V}$ battery, $6\units{\Omega}$ resistor, $5\units{\mu F}$ capacitor, and $0.5\units{mH}$ inductor.
\subimport*{Images}{P20img1.tex}
If the switch has been closed for a long time and is suddenly opened, find:
\begin{enumerate}
    \item How much current is flowing through the inductor just after the switch has been opened.
    \item How much charge is stored on the capacitor at the instant the switch is opened.
    \item The initial energy in the circuit.
    \item The max current in the circuit.
    \item How often the capacitor attains a max charge per second (only consider the absolute value of the charge).
    \item How much power is dissipated in the circuit after 40 seconds.
\end{enumerate}

\iftoggle{solutions}{
\textbf{Solution:}
\begin{enumerate}
    \item When the switch is closed for a long time the capacitor is fully charged and all the current is flowing through the inductor. This current can be computed as:
    \[i_L(0^-)=\frac{\varepsilon}{R}=\frac{12}{6}=2\units{A}\]
    The instant the switch is opened, the current through the inductor will want to stay the same so the current will be
    \[i(0^+)=i(0^-)=\answer{2\units{A}}\]
    \item The capacitor was fully charged at the moment the swicth was open so the charge will be
    \[q(0^+)=q(0^-)=\varepsilon C=(12)(5\cdot10^{-6})=\answer{60\units{\mu C}}\]
    \item The total energy stored in the new circuit will be the initial enery stored in the capacitor plus the initial energy stored in the inductor.
    \begin{align*}
        &U_C=\frac{1}{2}C\varepsilon^2=\frac{1}{2}(5\cdot10^{-6})(12)^2=0.36\units{mJ}\\
        &U_L=\frac{1}{2}Li^2=\frac{1}{2}(0.5\cdot10^{-3})(2)^2=1\units{mJ}\\
        &U_\text{total}=U_C+U_L=\answer{1.36\units{mJ}}
    \end{align*}
    \item The max current will occur when all of the energy is stored in the inductor.
    \begin{align*}
        &U_\text{total}=U_L=\frac{1}{2}Li_\text{max}^2\\
        &i_\text{max}=\sqrt{\frac{2U_\text{total}}{L}}=\answer{2.33\units{A}}
    \end{align*}
    \item We can solve this by using the angular frequency of the circuit which is defined to be
    \begin{align*}
        &\omega=\frac{1}{\sqrt{LC}}=2\pi f\\
        &f=\frac{1}{2\pi\sqrt{LC}}=3183\units{Hz}
    \end{align*}
    Note that this is the frequency for which the charge will attain a positive maximum. We must also account for when the current switches directions and we get a negative maximim so our frequency will be double this value.
    \begin{align*}
        &f_\text{abs}=6366\units{Hz}\\
        &N=(6366\units{Hz})\cdot(1\units{s})=\answer{6366\text{ times}}
    \end{align*}
    \item This is a trick question. Power is only dissipated in an RLC circuit through resistors. Because there are no resistors connected, there will be \underline{no power dissipated}.
\end{enumerate}
}{}