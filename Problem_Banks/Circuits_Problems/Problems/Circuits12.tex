\subsection*{Problem 12}
\iftoggle{contributions}{
\textit{2023 PHYS 158 Homework 3 Question 6}
}{}

\iftoggle{difficulty}{
Difficulty: $\medblackstar \medblackstar \medwhitestar$
}{}

The circuit shown below initially has no charge on the capacitors and the switch S is
originally open. Use $R_1 = 4\units{\Omega}$, $R_3 = 8\units{\Omega}$, $R_4 = 8\units{\Omega}$, $R_5 = 6\units{\Omega}$ $C_1 =2\units{\mu F}$, and $C_2 = 6\units{\mu F}$.\\
\centerline{\includegraphics[width=0.5\textwidth]{Images/P12img1.png}}
\begin{enumerate}
    \item Just after closing the switch S, find the currents $I_1$, $I_2$, $I_3$, $I_4$, $I_5$.
    \item After the switch has been closed for a very long time, find the currents $I_1$, $I_2$, $I_3$, $I_4$, $I_5$.
    \item After the switch S has been closed for a very long time, find the potential at points A, B, C, D, E, \& F.
\end{enumerate}
(Note that the directions of the arrows don't necessarily mean that the current will be flowing in that direction. Also assume that no current flows to ground and that it just serves as a 0 reference point for the voltage.)

\iftoggle{solutions}{
\textbf{Solution:}\\
$R_1=4\Omega,\;R_3=8\Omega,\;R_4=8\Omega,\;R_5=6\Omega.\;C_1=2\mu F,\;C_2=6\mu F$ at $t=0$: $q_C(0)=0$ for $C_1$ and $C_2$
\vspace{5mm}

\begin{enumerate}
	\item At $t=0$, $V_{C_1}=0$ and $V_{C_2}=0$ since $q_i(0)=0\implies$ both capacitors can be replaced by wires.
	
	\vspace{5mm}
	\includegraphics[width=0.55\linewidth]{Images/P12img2.png}
	\vspace{5mm}
	
	Note $V_A=V_E=V_B$ (as they are connected by wires), $R_1$ and $R_4$ are in parallel
	$$\implies \frac{1}{R_{eq}}=\frac{1}{R_1}+\frac{1}{R_4}=\frac{1}{4}+\frac{1}{8}=\frac{3}{8}$$.
	
	Hence $R_{eq}=\frac{8}{3}=2.67A$, also $I_5=I_1+I-2+I_3$ but \underline{$I_3=0$} (shorted)
	\vspace{5mm}
	
	Note that $I_4=I_2+I_3$ and $I_5 = I_1+I_4$, $V_{EF}=V_{AC}$ so $I_4R_4 = I_1R_1$
	\vspace{5mm}
	
	This means that $12V=I_4R_4+I_5R_5=I_1R_1+I_5R_5$
	\vspace{5mm}
	
	Hence $12V=8I_4 + 6(2I_4+I_4)=I_4(8+12+6)=26I_4$
	\begin{align*}
		&\implies I_4=0.46A\textrm{ (down)}\\
		&\implies I_1=0.92A \textrm{ (to right)}\\
		&\implies I_2=I_4=0.46A \textrm{ (to right)}\\
		&\implies I_5 = 1.38A \textrm{ (to right)}\\
	\end{align*}
	\vspace{5mm}
	
	\item After a long time $I_C\rightarrow 0$, $Q_{C_1}=V_{C_1}(C_1)$, and $Q_{C_2}=V_{C_2}(C_2)$. Additionally, the Capacitors will no longer allow charge to increase so $I_{C_1}=I_{C_2}=0\rightarrow$ \underline{an open circuit.} 
	
	Redraw the circuit $\implies$
	
	\vspace{5mm}
	\includegraphics[width=0.55\linewidth]{Images/P12img3.png}
	\vspace{5mm}
	
	Note \underline{$I_2'=I_1'=0$} and \underline{$I_5'=I_3'=I_4'$}
	\vspace{5mm}
	
	Hence:
	$$12V=I_5'(R_3+R_4+R_5)=I_5'(8+8+6)=22I_5'$$
	$$\underline{I_5'=\frac{12}{22}A=0.545A}$$
	
	
	\item $\underline{V_A=V_B=12\textrm{Volts}}$, and we have:
	
	$$12-0.545(8)=12-0.436=7.64V$$
	$$V_E=12-I_3'R_3=12-0.55(8)=12-4.4=7.6V$$
	$$V_E=0+I_5'R_5+I_4'R_4=I_5'(R_4+R_5)=0.545(6+8)=0.545(14)=\underline{V_E=7.63V}$$
	$$V_C=V_F=I_5'R_5=0.545(6)=\underline{3.27V=V_C=V_F}$$
\end{enumerate}
}{}