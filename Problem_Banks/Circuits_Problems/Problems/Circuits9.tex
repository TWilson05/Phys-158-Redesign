\iftoggle{title}{
\subsection*{Problem 9}
}{}
\iftoggle{contributions}{
\textit{2023 PHYS 158 Homework 3 Question 3}
}{}

\iftoggle{difficulty}{
Difficulty: $\medblackstar \medblackstar \medblackstar$
}{}

A light bulb is connected in the circuit shown below -- the switch S is initially open and the
capacitor is is uncharged. The battery has no appreciable internal resistance and a voltage = $12\units{V}$. The
resistance of the light bulb is $1000 \units{\Omega}$, $R = 50\units{\Omega}$ and $C = 100\units{\mu F}$ . The switch is closed at $t=0$.\\
\centerline{\includegraphics[width=0.3\textwidth]{Images/P9img2.png}}
\begin{enumerate}
    \item Find the all of the currents at $t=0^+$.
    \item Find all of the currents after a very long time.
    \item Find the current in the bulb as a function of time.
    \item Sketch the brightness of the bulb as a function of time.
\end{enumerate}

\iftoggle{solutions}{
\textbf{Solution:}\\
Light Bulb circuit:
\vspace{5mm}

\includegraphics[width=0.35\linewidth]{Images/P9img2.png}

$V=12\textrm{Volts}$, $R_B=1000\Omega$, $R=50\Omega$, $C=100\mu F$


$$K_1 \implies I = I_B+I_C$$
$$K_2 \implies V-IR-\frac{q_C}{C}=0$$
$$V-IR-I_BR_B=0$$

\begin{enumerate}
	\item  At $t=0$ $q_C=0$ so $V_C=0$ and $V_C=V_B=0=I_BR_B$. Hence $I_B=0$ and $\underline{I=I_C=\frac{V}{R}=\frac{12}{50}\textrm{A}}=0.24\units{A}$.
	\vspace{5mm}
	
	\item As $t\rightarrow \infty$, $I_C\rightarrow 0$ so $I=I_B=\frac{V}{R+R_B}=\underline{\frac{12}{1050}A}=11.43\units{mA}$.
	\vspace{5mm}
	
	\item Now we study the time dependence of $I,I_B,I_C$. Consider the following equation from earlier $V-IR-q_C/C=0$
	$$\frac{d}{dr}\left(V-IR-\frac{1}{C}{q_C}\right)=0=0-R\frac{dI}{dt}-\frac{1}{C}\frac{dq}{dt}.$$
	
	Hence
	
	$$\frac{dq_C}{dt}=\underline{I_C=-RC\frac{dI}{dt}}.$$
	
	Consider the following equation from earlier $I_C=I-I_B$ so $-RC\frac{dI}{dt}=I-I_B$
	
	and finally use the equation $I_B=\frac{V-IR}{R_B}$ so that $$-RC\frac{dI}{dt}=I-\frac{1}{R_B}(V-IR)=\frac{-V}{R_B}+I(1+\frac{R}{R_B}).$$
	
	Simplifying $\implies -RR_BC\frac{dI}{dt}=-V+I(R_B+R)$ or 
	
	$$\frac{dI}{dt}=\frac{V}{RR_BC}-\frac{I}{C}\left(\frac{R_B+R}{R_BR}\right)$$
	
	$$\underline{\frac{dI}{dt}=\frac{V}{RR_BC}-\frac{I}{C} \left(\frac{1}{R}+\frac{1}{R_B}\right)}$$
	
	this has a solution $I(t)=A\exp(-t/\tau)+B$ so
	
	$$\frac{dI}{dt}=-\frac{A}{\tau}\exp(-t/\tau)$$
	
	substituting and solving for A and B we find (using) $I(0)=\frac{V}{R}$
	$$\implies A= \left(\frac{V}{R}-\frac{V}{R+R_B}\right), \: B=\left(\frac{V}{R+R_B}\right)$$
	
	$$\therefore I(t)=V\left(\frac{1}{R}-\frac{1}{R+R_B}\right)\exp(-t/\tau)+\frac{V}{R+R_B}$$
	
	$$\underline{I(t) = \frac{VR_B\exp(-t/\tau)}{R(R+R_B)}+\frac{V}{R+R_B}}$$
	
	$$\rightarrow \tau=\frac{RR_BC}{R+R_B} \textrm{ from found equation for rate of change of current.}$$
	
	Finally
	$$I_C(t)=I(t)-I_B(t),\textrm{ but } I_B(t)=\frac{V-IR}{R_B}.$$
	$$\underline{I_B(t)=\frac{V}{R_B}-\frac{R}{R_B}\left(\frac{V}{R+R_B}\left(1+R_B\exp(-t/\tau\right)\right)}$$
	
	Inserting the values $\tau=4.76ms, \: \frac{1}{\tau}=210s^{-1}$
	\vspace{5mm}
	
	$I_C(t)=240\exp(-210t)mA$
	
	$I_B(t)=11.43mA(1-\exp(-210t))$
	
	\item See sketch below
	
	\vspace{5mm}
	\includegraphics[width=0.7\linewidth]{Images/P9img3.png}
	
	
	$P_B=I_B^2R_B$
	
	\vspace{5mm}
	\textbf{\large This problem is too difficult for any EXAM!!}
\end{enumerate}
}{}