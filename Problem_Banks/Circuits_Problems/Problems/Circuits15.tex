\subsection*{Problem 15}
\iftoggle{contributions}{
\textit{Created by Tyler Wilson 2023}
}{}

\iftoggle{difficulty}{
Difficulty: $\medblackstar \medwhitestar \medwhitestar$
}{}

For the following resistor configuration, find the equivalent resistance.\\
$R_1=2\units{\Omega}$, $R_2=8\units{\Omega}$, $R_3=3\units{\Omega}$, $R_4=10\units{\Omega}$, $R_5=12\units{\Omega}$, $R_6=7\units{\Omega}$, $R_7=4\units{\Omega}$.
\subimport*{/Images}{P15img1.tex}

\iftoggle{solutions}{
\textbf{Solution:}\\
We can recognize that $R_2$ and $R_3$ are in parallel and so we can compute the equivalent as
\[\frac{1}{R_{23}}=\frac{1}{R_2}+\frac{1}{R_3}=\frac{1}{8}+\frac{1}{3}=\frac{11}{24}\Ra R_{23}=\frac{24}{11}\units{\Omega}=2.18\units{\Omega}\]
$R_4$ and $R_5$ are also in parallel and can be combined
\[\frac{1}{R_{45}}=\frac{1}{R_4}+\frac{1}{R_5}=\frac{1}{10}+\frac{1}{12}=\frac{11}{60}\Ra R_{54}=\frac{60}{11}\units{\Omega}=5.45\units{\Omega}\]
We can then recognize that $R_{45}$ and $R_6$ are in series and we can compute
\[R_{456}=R_{45}+R_6=\frac{60}{11}+7=\frac{137}{11}\units{\Omega}=12.45\units{\Omega}\]
Then we get that $R_{23}$ and $R_{456}$ are in parallel and get
\[\frac{1}{R_{23456}}=\frac{1}{R_{23}}+\frac{1}{R_{456}}\Ra R_{23456}=\frac{1}{\frac{11}{24}+\frac{11}{137}}= 1.86\units{\Omega}\]
The remaining resistors will all be in series which gives
\[R_\text{eq}=R_1+R_{23456}+R_7=2+1.86+4=\answer{7.86\units{\Omega}}\]
}{}