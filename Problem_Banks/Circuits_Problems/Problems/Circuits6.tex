\iftoggle{title}{
\subsection*{Problem 6}
}{}
\iftoggle{contributions}{
\textit{Created by Tyler Wilson 2023}
}{}

\iftoggle{difficulty}{
Difficulty: $\medblackstar \medblackstar \medwhitestar$
}{}

Using their newfound knowledge of LR circuits, a Phys 158 student came up with a clever idea for a prank. They want to design an alarm clock that will continue to ring for 10 seconds after the battery is removed. The alarm clock can be thought of as a $1\units{k\Omega}$ resistor which requires at least 1 Watt to operate. They designed the following circuit to achieve this.
\subimport*{Images/}{P6img1.tex}
\begin{enumerate}
\item What value should the battery be such that the power supplied to the alarm clock does not exceed 3 Watts?
\item What value of inductor should they use so that the alarm clock remains on for 10 seconds after the battery is disconnected?
% \item Plot the energy stored in the inductor as a function of time after the batter has been disconnected.
\end{enumerate}

\iftoggle{solutions}{
\textbf{Solution:}
\begin{enumerate}
\item For maximum power to be supplied to the clock, $P_c=3\units{W}$
\begin{align*}
    &P_C=i_C^2R_C\Ra i_C=\sqrt{\frac{P_C}{R_C}}\\
    &\varepsilon=V_C=i_CR_C=\sqrt{\frac{P_C}{R_C}}\cdot R_C=\sqrt{P_CR_C}=\sqrt{(3\units{W})(1000\units{\Omega})}=54.8\units{V}
\end{align*}
\item We can start by writing out Kirchoff's loop voltage law for the circuit and then solving the resulting differential equation by separation of variables to get an expression for the current as a function of time:
\subimport*{Images/}{P6img2.tex}
\begin{align*}
    &iR+iR_C+L\ddx[i]{t}=0\\
    &L\ddx[i]{t}=-i(R+R_C)\\
    &\ddx[i]{t}=-\frac{R+R_C}{L}i\\
    &\frac{di}{i}=-\frac{R+R_C}{L}dt\\
    &\int\frac{di}{i}=-\frac{R+R_C}{L}\int dt\\
    &\ln|i|=-\frac{R+R_C}{L}t+\text{Constant}\\
    &i(t)=e^{-\frac{R+R_C}{L}t+\text{Constant}}=i_0e^{-\frac{R+R_C}{L}t}
\end{align*}
Alternatively, we can get the same expression by thinking about it conceptually and computing the time constant.\\
We know that the current will initially want to stay the same because of the inductor and it will slowly decay to 0 so we know that the equation of the current should look like exponential decay and be of the form
\[i(t)=i_0e^{-t/\tau}\]
We can then compute the time constant for an RL circuit as
\[\tau=\frac{L}{R_{eq}}=\frac{L}{R+R_C}\]
Plugging this in will yield the same expression as above.\\
The initial current, $i_0$, will be the current that was initially flowing through the inductor. We computed this in part (a) to be
\[i_0=i_C=\sqrt{\frac{P_C}{R_C}}=\sqrt{\frac{3\units{W}}{1000\units{\Omega}}}=54.8\units{mA}\]
Now we have a complete expression for the current as a function of time. We can get the power as a function of time as
\begin{align*}
    &P(t)=i^2R_C=i_0^2R_Ce^{-\frac{2(R+R_C)}{L}t}\\
    &i_0^2R_C=\frac{P_C}{R_C}\cdot R_C=P_C\\
    &P(t)=P_Ce^{-\frac{2(R+R_C)}{L}t}
\end{align*}
We are told that the clock must have a minimum of 1 Watt and we want it to last for 10 seconds so we can set $P=1$ and $t=10$ and solve for $L$.
\begin{align*}
    &\frac{P}{P_C}=e^{-\frac{2(R+R_C)}{L}t}\\
    &\ln\bfrac{P}{P_C}=-\frac{2(R+R_C)}{L}t\\
    &L=-\frac{2(R+R_C)t}{\ln\bfrac{P}{P_C}}=-\frac{2(100\units{\Omega}+1000\units{\Omega})(10\units{s})}{\ln\bfrac{1\units{W}}{3\units{W}}}=20,025\units{H}
\end{align*}
\end{enumerate}
}{}