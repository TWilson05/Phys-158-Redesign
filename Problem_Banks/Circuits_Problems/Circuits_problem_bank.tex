\documentclass[11pt, fleqn]{article}
\usepackage[utf8]{inputenc}
\usepackage{fullpage}
\usepackage{amsmath,amssymb,array}
\usepackage{dsfont}
\usepackage{amsfonts}
\usepackage{graphicx}
\usepackage{mathtools}
\usepackage{polynom}
\usepackage{esint}
\usepackage{mathrsfs}
\usepackage{fizztex}
\usepackage{import}
\usepackage{tikz}
\usepackage{circuitikz}
\usepackage{etoolbox} % included for conditionals
% use to make stars for difficulty
\usepackage{stix}
\setlength{\parindent}{0em}

% sets the enumerate to a), b), c), ...
\renewcommand{\labelenumi}{\alph{enumi})}

% define conditionals
\newtoggle{solutions}
\newtoggle{contributions}
\newtoggle{difficulty}

% set conditionals values
\toggletrue{solutions}
\toggletrue{contributions}
\toggletrue{difficulty}
% \togglefalse{solutions}
% \togglefalse{contributions}
% \togglefalse{difficulty}

\title{Physics 158 Circuits Problem Bank}
\author{}
\date{}

\begin{document}
\allowdisplaybreaks

\maketitle

\section*{Time Independent Circuits}
\import{Problems/}{Circuits15.tex}
\import{Problems/}{Circuits16.tex}
\import{Problems/}{Circuits17.tex}

\section*{Time Dependent Circuits}
\import{Problems/}{Circuits1.tex}
% \import{Problems/}{Circuits2.tex} % used in tutorial
\import{Problems/}{Circuits3.tex}
\import{Problems/}{Circuits4.tex}
\import{Problems/}{Circuits5.tex}
\import{Problems/}{Circuits6.tex}
\import{Problems/}{Circuits7.tex}
\import{Problems/}{Circuits8.tex}
\import{Problems/}{Circuits9.tex}
\import{Problems/}{Circuits10.tex}
\import{Problems/}{Circuits11.tex}
\import{Problems/}{Circuits12.tex}

\section*{LC and RLC Circuits}
\import{Problems/}{Circuits20.tex}
% \import{Problems/}{Circuits21.tex} % no solution

\section*{AC Circuits}
\import{Problems/}{Circuits13.tex}
\import{Problems/}{Circuits14.tex}
\import{Problems/}{Circuits18.tex}
\import{Problems/}{Circuits19.tex}

\end{document}
