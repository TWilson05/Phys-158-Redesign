\iftoggle{title}{
\subsection*{Problem 11}
}{}
\iftoggle{contributions}{
\textit{2023 PHYS 158 Homework 6 question 3}
}{}

\iftoggle{difficulty}{
Difficulty: $\medblackstar \medblackstar \medwhitestar$
}{}

Consider a plastic ring of radius $R = 50.0 \units{cm}$ on which there are two charged beads as shown in the figure below.
Bead 1 has $q_1 = +2.00 \units{\mu C}$ and is fixed in place on the x-axis. Bead 2 has $q_2 = +6.0 \units{\mu C}$ and can be moved along the ring.
\begin{enumerate}
    \item Determine the positive angle $\theta$ for $q_2$ which can produce a net Electric field of magnitude $E = 2.00 \times 10^5 \units{N/C}$ at the centre of the ring?
    \item Is there any negative angle $\theta$ for $q_2$ which can produce a net Electric field of
    magnitude $E = 2.00 \times 10^5 \units{N/C}$ at the centre of the ring? If so, calculate that angle.
\end{enumerate}

\begin{center}
    \begin{tikzpicture}
        \draw[fill=black] (0,0) circle (0.05);
        \draw (0,0) -- (45:2) node[midway, above left] {$R$};
        \draw[fill=black] (45:2) circle (0.1) node[above right] {Bead 2};
        \draw (0,0) circle (2);
        \draw (0.5,0) arc (0:45:0.5) node[midway, above right] {$\theta$};
        \draw[fill=black] (-2,0) circle (0.1) node[above left] {Bead 1};
        \draw (-3,0) -- (3,0) node[right] {$x$};
        \draw (0,-3) -- (0,3) node[above] {$y$};
    \end{tikzpicture}
\end{center}

\iftoggle{solutions}{
\textbf{Solution:}
\begin{center}
    \includegraphics[width=\textwidth]{Images/P11img1.png}
    \includegraphics[width=\textwidth]{Images/P11img2.png}
\end{center}
}{}