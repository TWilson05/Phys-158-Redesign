\iftoggle{title}{
\subsection*{Problem 13}
}{}
\iftoggle{contributions}{
\textit{2023 PHYS 158 Homework 6 question 5}
}{}

\iftoggle{difficulty}{
Difficulty: $\medblackstar \medblackstar \medwhitestar$
}{}

An $\alpha$ particle is a positively charge particle which is the nucleus of a helium atom 4He. An engineer who studied
Physics 158 E\&M is designing a detector using a long metal wire and a uniform plastic plate. The details are as follows\\
-- the long line carrying a uniform linear charge density $+50.0 \units{\mu C/m}$ runs parallel to and $10.0\units{cm}$ from the surface of a
large, flat plastic plate that has a uniform surface charge density of $-100\units{\mu C/m^2}$ on one side.
\begin{enumerate}
    \item Determine the individual Electric Fields produced by the wire and the plate.
    \item Determine the total Electric Field produced by the wire and plate.
    \item Determine the locus of all points where an $\alpha$ particle would feel no force due to this arrangement of
    charged objects. Assume the plastic plate lies in the (x,y) plane.
\end{enumerate}

\iftoggle{solutions}{
\textbf{Solution:}
\begin{center}
    \includegraphics[width=\textwidth]{Images/P13img1.png}
    \includegraphics[width=\textwidth]{Images/P13img2.png}
\end{center}
}{}