\subsection*{Problem 3}
\iftoggle{contributions}{
\textit{Created by Tyler Wilson 2023}
}{}

\iftoggle{difficulty}{
Difficulty: $\medblackstar \medwhitestar \medwhitestar$
}{}

After learning about magnetic induction, a Physics 158 student decides to try to make their own generator as a personal project. They create a wire loop with $N$ coils and a radius of 0.1 meters. The wires are connected to a light bulb which has a resistance of 100 $\Omega$ and requires a power of 2 Watts to light up. Inside the wire loop is a magnet with a magnetic field of 0.1 T. The magnet is fixed on a shaft that rotates at a speed $\omega$. 
\begin{enumerate}
    \item If 500 coils are used, what is the minimum speed the magnet needs to be spinning in order to briefly light up the light bulb?
    \item The student now wants to hook up this application to a bicycle. Assume that the bike tires have a radius of 17 cm and that the generator is set up with a gear ratio such that every rotation of the bike tire corresponds to 7 rotations of the magnet. If the student is able to maintain a constant speed of 10 km/h, what is the minimum number of coils needed such that the average power from one revolution of the magnet is enough to light up the light bulb?
\end{enumerate}



\begin{center}
    \begin{tikzpicture}
        \draw (0,0) circle (2);
        \draw (0,0.05) circle (2);
        \draw (0,0.1) circle (2);
        \draw (-0.5,-1) rectangle (0.5,1);
        \draw[red, ->] (0,-1) -- (0,1) node[above] {$\vec{B}$};
        \draw[red, ->] (0.4,-1) -- (0.4,1);
        \draw[red, ->] (-0.4,-1) -- (-0.4,1);
        \draw[dashed] (-3,0) -- (-0.5,0);
        \draw[dashed] (0.5,0) -- (3,0);
        \draw[postaction={decorate, decoration={
            markings,
            mark=at position 0.2 with {\arrow{>}}
        }}] (3,0) ellipse (0.12 and 0.25);
        \node[right] at (3.12,0) {$\omega$};
        \draw[->] (3,1) -- +(1,0) node[right] {$\hat{i}$};
        \draw[->] (3,1) -- +(0,1) node[above] {$\hat{j}$};
    \end{tikzpicture}
\end{center}

\begin{center}
\begin{circuitikz}
    \draw (0,0)
    to[sV, l_=Generator] (0,-2)
    to[short] (2,-2)
    to[R, l_=$100\units{\Omega}$] (2,0)
    to[short] (0,0);
\end{circuitikz}
\end{center}

\iftoggle{solutions}{
\textbf{Solution:}
\begin{enumerate}
    \item We can compute the emf induced in the loop by using Faraday's law.
    \begin{align*}
        &\varepsilon=-\ddx[\Phi_B]{t}
    \end{align*}
    The magnetic flux through the loop is given by
    \[\Phi_B=\vec{A}\cdot\vec{B}=AB\cos\theta\]
    Note that in this setup the magnetic field and the area of the loops are both constant and we can take them out of the derivative.
    \begin{align*}
        &\varepsilon=-\ddx[\Phi_B]{t}=\ddx{t}(AB\cos\theta)=-AB\ddx{t}\cos\theta
    \end{align*}
    Note that the derivative of $\theta$ is equal to the angular velocity of the magnet.
    \begin{align*}
        &\varepsilon=-AB\ddx{t}\cos\theta=-AB\ddx{\omega}{t}=AB\sin\theta\ddx{\theta}{t}=AB\omega\sin\theta
    \end{align*}
    We can express the area as the area of a circle with radius $r$ multiplied by the number of coils.
    \begin{align*}
        &A=\pi r^2N\\
        &\varepsilon=AB\omega\sin\theta=\pi r^2NB\omega\sin\theta
    \end{align*}
    The power generated for this circuit can be computed as
    \begin{align*}
        &P=\frac{\varepsilon^2}{R}=\frac{\pi^2r^4N^2B^2\omega^2\sin^2\theta}{R}
    \end{align*}
    Rearranging for $\omega$ we get
    \begin{align*}
        &\omega=\frac{\sqrt{PR}}{\pi r^2NB\sin\theta}
    \end{align*}
    Depending on the value of $\theta$ (which point the magnet is at in its rotation) we will get a different value for $\omega$. We want to find the minimum value of $\omega$ so we can take the maximum value of $\sin\theta$ which is 1. This gives us the minimum value of $\omega$ as
    \[\omega_{min}=\frac{\sqrt{PR}}{\pi r^2NB}=\frac{\sqrt{2\cdot100}}{\pi(0.1)^2(500)(0.1)}\approx\answer{9.00\units{rad/s}}\]

    \item We can start by finding the angular velocity of the bike tire.
    \begin{align*}
        &v=\frac{10\units{km}}{\units{h}}\cdot\frac{1\units{h}}{3600\units{s}}\cdot\frac{1000\units{m}}{1\units{km}}=2.\bar{7}\units{m/s}\\
        &\omega_\text{tire}=\frac{v}{r}=\frac{2.\bar{7}\units{m/s}}{0.17\units{m}}\approx16.3\units{rad/s}
    \end{align*}
    We are told that every rotation of the bike tire corresponds to 7 rotations of the magnet so we can find the angular velocity of the magnet as
    \begin{align*}
        &\omega_\text{magnet}=7\omega_\text{tire}\approx114.4\units{rad/s}
    \end{align*}
    We can now use the same equation as in part (a) to compute the average power generated by the magnet.
    \begin{align*}
        &P=\frac{\pi^2r^4N^2B^2\omega^2\sin^2\theta}{R}\\
        &P_{avg}=\int_0^{2\pi}\frac{\pi^2r^4N^2B^2\omega^2\sin^2\theta}{R}d\theta=\frac{\pi^2r^4N^2B^2\omega^2}{R}\int_0^{2\pi}\sin^2\theta d\theta=\frac{\pi^2r^4N^2B^2\omega^2}{R}\pi\\
    \end{align*}
    We can then rearrange for $N$ to find the minimum number of coils needed.
    \begin{align*}
        &N=\sqrt{\frac{P_{avg}R}{\pi^3r^4B^2\omega^2}}=\sqrt{\frac{2\cdot100}{\pi^3(0.17)^4(0.1)^2(114.4)^2}}\approx7.68\units{coils}
    \end{align*}
    Rounding up to the nearest whole number, we get that we require $\answer{8}$ coils.
\end{enumerate}

}{}