\subsection*{Problem 17}
\iftoggle{contributions}{
\textit{Created by Tyler Wilson 2023}
}{}

If the total power dissipated in the circuit is 15W, what is the value of $R$?
\subimport*{Images/}{P17img1.tex}

\iftoggle{solutions}{
\textbf{Solution:}\\
The total power dissipated in the circuit will be the sum of the power dissipated in each resistor. We know that the power dissipated across a resistor is $P=IV$. We also know $V=IR$ and can rearrange to get $P=\frac{V^2}{R}$.\\
The voltage across the $8\units{\Omega}$ resistor is $8\units{V}$ so power across that resistor will be
\[P_1=\frac{(8\units{V})^2}{8\units{\Omega}}=8\units{W}\]
The sum of the power across the remaining two resistors will be $P_\text{total}-P_1=15\units{W}-8\units{W}=7\units{W}$.\\
We can combine resistor $R$ and the $6\units{\Omega}$ to get an equivalent resistor with value $R+6$. The volatge across this equivalent resistor will be $8\units{V}$ and will have a power usage of $7\units{W}$. We can then solve for $R$ as
\begin{align*}
    &7\units{W}=\frac{(8\units{V})^2}{R+6\units{\Omega}}\\
    &R=\frac{(8\units{V})^2}{7\units{W}}-6\units{\Omega}=\answer{3.14\units{\Omega}}
\end{align*}
}{}